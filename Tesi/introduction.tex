\chapter{Introduction}
\label{cha:introduction}
Wikipedia is the largest encyclopedia available on the Internet. We could argue that it is the largest collection of knowledge ever created in the entire human history. No other library or oral knowledge could ever stand the comparison in size. It is also important to note that Wikipedia is available everywhere, for anyone in almost any language. Wikipedia, in fact, makes the knowledge accessible to anyone, in a democratic way. This is a remarkable way of spreading wisdom.\\
For all the above reasons is fair to say that Wikipedia has a tremendous importance in our modern society. Such an important source of knowledge requires an adequate maintenance and care, in order to ensure its safety and its prosperity.
\section{Context and motivation}
\label{sec:context}
Everybody deserves the possibility to learn, to know, to explore the human knowable. But in order to provide this amazing opportunity to whoever asks for, there must be someone who takes care of it. \\
This is the Wikipedia trade-off: if something is open and reachable by everyone, it must be provided and taken care of by many people. In order to ensure the democracy of the project, everyone must join it.\\
Without its fundamental active users, its contributors, Wikipedia ceases to be what it is now.\\
There cannot be readers without writers.\\
Driven by these very motivations, the Wikimedia Community Health Metrics projects aims at taking care of the online encyclopedia, collect information about it and spread them to share knowledge among Wikipedians.
\section{Project description}
\label{sec:project}
The goal of the Wikimedia Community Health (WikiCHM) project is to measure and understand the Wikipedia community health status \cite{consonni}. The WikiCHM project wants to monitor it, because if we can understand something, we can interact with it, we can change it, and hopefully, we can improve it. In fact, the project does not limit itself to an ephemeral comprehension of the wellness of Wikipedia. It wants to give recommendation in order to ensure its prosperity and spread the information about it. These tips are for everybody who interacts with Wikipedia but are especially meant for its active users. \\
In order to make this data understandable by everyone, even by non-technical people, visual dashboards were created. These graphs make the information about Wikipedia health easy to read and its comprehension straight forward.\\
But before measuring Wikipedia health, we must define it.
Like doctors can measure human body vital signs (heartbeat, cerebral activity, oxygen levels,...), so can we too: the project defines six metrics that can be analyzed and studied. These are called \textit{Wikipedia Vital Signs}. \\

\section{Vital Signs}
\label{sec:research}
Vital signs are well-defined concepts that cast the light on different aspects of the Wikipedia community. \\
Thanks to the contribute of each vital sign, we can determine the overall health status of Wikipedia. More precisely, we will measure the vital signs of each language community. Since Wikipedia is the sum of all its parts, we can understand its general wellness measuring the health status of each small part.\\
The project envisions six vital signs that will be monitored:
\pagebreak
\begin{itemize}
    \item Retention
    \item Stability
    \item Balance
    \item Special functions
    \item Admin flags
    \item Global community participation
\end{itemize}
These metrics, defined by the Eurecat project team, provide a powerful description of the Wikipedia communities health status.\\
The first three are about the Wikipedia active users: the editors. While the last three are about more technical aspects. \\
We are given multiple indicators to have a better understanding of different important aspects of the Wikipedia community.\\
\section{Visualizations}
\label{sec:product}
The vital signs indicators are accurately and individually shown in the interactive dashboards. \\
Each vital sign has (at least) one peculiar chart that shows how its peculiar values changed over time. This is what we are really interested in: to visualize and understand how the metrics evolved throughout Wikipedia life. \\
Depending on the metric, our attention will be directed towards the rise or the fall of a certain values. Is it good that this particular metric increased? Should it have decreased? These questions find their answers in the highlights. We can find them right under the graphs, they help the user to understand better the visualized metric.\\
Both the graphs and the highlights are dynamic: the dashboards present some interactive components that allow the user to change different aspects of the visualizations and of the highlights:
\begin{itemize}
    \item \textbf{Language}: Since we want to study the several Wikipedia language communities, the dashboards allow the user to select among every language available on Wikipedia. The user then obtain one chart for each selected language, to let comparison between different language happen.
    \item\textbf{Time}: We want to understand the evolution in time of the Wikipedia health status. In fact, we can change the time aggregation of the data from yearly to monthly, in order to have a general or accurate view. We can also select the time period analyzed through an intuitive window that can slide left or right, allowing to scroll through the years (or month).
    \item\textbf{Data}: For each vital sign there are peculiar parameter that can be selected, in order to let the user obtain the right chart with the right type of desired data. The ultimate goal is to spread knowledge and understanding.
    \item\textbf{Value}: Finally, we can decide if we want to see absolute or percentage values, to fulfill every research task.
    Absolute values are more useful if we want to conduct a quantitative analysis on singular language communities.\\
    Percentage values should be preferred in case of a qualitative analysis to compare different language communities.\\
\end{itemize}
Every graph is also interactive as well: when passing the cursor over the figures a little text box appears, with useful information about the specific selected coordinates. It is also possible to slide a window left or right, to analyze a specific time period and let the user decide how long it should be.\\
Everything is immediate and intuitive to provide the best user experience while letting the user to study and understand the Wikipedia health metrics.     \\
In the next chapters, we will get a closer look at these visual dashboards: how they work and which conclusions we can infer from them.
\\




