\chapter{Conclusions}
\label{cha:conclusions}

As described in the previous chapter, the goal of this thesis was to build interactive online dashboards to allow quantitative and qualitative analysis to be done on the health status of the Wikipedia community, expressed under the sum of each contribution coming from every language community present in the online encyclopedia.
\pagebreak
\paragraph{Results}

The dashboards are meant to be comprehensible by everyone, not only for data scientist and researchers. The WikiCHM project aims at spreading knowldge and awareness among the Wikipedians, in order to inspire them at acting in a way to improve the six community vital signs. For this very goal, the visualizations have been made minimal, easy to interact with and to understand. Particular attention has been made to the user experience to prevent unnecessary page reloading every time a parameter is changed, thus the use of asynchronous callback and the implementation of an interactive layout. Furthermore, the automatic highlights make the charts even more human-friendly.\\
The dashboards help us to read data that would be otherwise hard to understand in a raw form of a database. \\
In general, we can conclude the pattern analyzed from these dashboards are the same for every major language. In fact, it is evident that since the birth of Wikipedia the trends are overall negative, since people are slowly drifting away from the online encyclopedia, risking to make it a static and immutable source. \\
We hope these dashboards can help solving this problem.

\paragraph{Contribution}

I have contributed at the project by

\begin{itemize}
    \item handling and analyzing over 20 GB of data from Wikipedia database 
    \item experimenting and discovering different kind of visualization and graphs for each health metric
    \item mainly, implementing and coding Python scripts to visualize and plot Wikipedia vital signs data
    \item studying different approaches for the code structure
    \item adapting and deploying the scripts from a local architecture to a remote server architecture
    \item studying how to improve the front-end user experience
    \item defining and generating automated description to enhance the graph 
\end{itemize}

I am personally very proud of having learned a new coding language, Python, and having realized a working and interactive website displaying visualizations about such an important topic as Wikipedia health metrics. \\
I hope that my small contribution to the this enormous project may help the community to improve as much as in its early days.

\paragraph{Future project steps}

The project does not stop here.
This was only the beginning of sharing knowledge among the Wikipedians in order to empower them. The goal is to make them realize they have the online enclyclopedia's future in their hands.\\
In the future, the Eurecat research team will provide useful Wikipedia pages about these topics, embedding the results in a table-like format, in order to make them accessible by everyone, on the Meta-wiki itself. The process will be automated, releasing the latest data each time they have them available.\\
The project will also publish and raise awareness on the existence of the dashboards, to make them a useful tool among editors and Wikipedians. The visualizations will always offer up to date charts.\\
The monitoring of Wikipedia health status will continue, by studying data and emerging and trends, and also by studying how the general public respond to the dashboards. Possibly, new visualizations will be realized or the existing ones will be corrected.
