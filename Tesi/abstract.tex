\chapter*{Abstract} % senza numerazione
\label{sommario}

\addcontentsline{toc}{chapter}{Abstract} % da aggiungere comunque all'indice

This dissertation describes in detail the activity performed during my two-month internship period at the Big Data \& Data Science Department of Eurecat – Centro Tecnológico de Catalunya, which was supervised by Cristian Consonni, Marc Miquel-Ribé and David Laniado.\\
The Community Health Metrics project aims at measuring and understanding the current Wikipedia health status, in order to bring awareness among the Wikipedia editors of the different communities.
Hence the realization of visual and interactive dashboards: since Wikipedia relies on its active editors, these dashboards are meant to be seen especially by them. It is important to render Wikipedia health status data in a way that is easy and straight forward, in order to make it understandable even by non-technical people.\\
The project wants to share knowledge in order to have a positive impact on Wikipedia health status.\\
I have contributed at the project by:

\begin{itemize}
  \item handling and analyzing over 20 GB of data from Wikipedia database 
    \item experimenting and discovering different kind of visualization and graphs for each health metric
    \item mainly, coding and creating Python scripts to visualize and plot Wikipedia Vital Signs’ data
    \item studying different approaches for the code structure
    \item adapting and deploying the scripts from a local architecture to a remote server architecture
    \item studying how to improve the front-end user experience
    \item defining and generating automated description to enhance the dashboards readability 
\end{itemize}
The database I analyzed was given to me by the Eurecat researchers: it is generated every month by observing the interactions of the Wikipedia editors in the different communities. Then, this data is grouped by the language of the communities, vital sign analyzed and other factors that will be later explained.\\
Given this database I created different Python scripts to generate the visual dashboards that graphically show the data taken from the database and the automatic captions, called highlights.\\
I implemented the visualization of the Wikipedia vital signs, six metric defined to study the health status of Wikipedia.\\
Each dashboard presented its unique challenge. From pre-elaborating the data in order to feed them correctly to the graph, from engineering the correct query statement, to retrieve the more comprehensible data, even to deciding which visualization to use to display a certain data class. \\
This dissertation discusses in details the implementation of each visualization and how it contributes to the dashboards, briefly mentioning the technologies employed and a minimal contextualization of the meaning of this metric. In the end, we will analyze the results emerged from these dashboards.


\pagebreak








